\documentclass{exam}
\title{MAT257 PSET 10---Question 3}
\author{Jonah Chen}
\date{}
\usepackage[utf8]{inputenc}
\usepackage[margin=0.5in]{geometry}

\usepackage{braket}
\usepackage{physoly}
\usepackage{currfile}
\usepackage{gensymb}
\usepackage{amssymb}
\usepackage{pgf,tikz,pgfplots}
\usepackage{mathrsfs}
\usepackage{textcomp}
\usepackage{parskip}
\setlength{\parindent}{0em}
\usetikzlibrary{arrows}
\numberwithin{equation}{section}
\pgfplotsset{compat=1.16}
\everymath{\displaystyle}
\newcommand{\R}{\mathbb{R}}

\begin{document}
    \sffamily
    \maketitle
    Let \(g(u,v)=(u/v,uv),\) and \(A=\{(u,v):1<u<\sqrt2,1<v<2\}.\) Then, \(g(A)=B\) and \(g\) has continuous partial derivatives except for \(v=0.\) Also, using mathematica \(\det(g')=2u/v\), which is nonzero except when \(u=0\). There are no points in \(A\) where either \(u\) or \(v\) is zero, hence, using the change of variables theorem,
    \[\int_Bx^2y^3=\int_A(u/v)^2(uv)^3=\int_Au^5v.\]

    As \(A\)'s boundary is a set of content zero, we can integrate over the closure of \(A,\overline A=[1,\sqrt2]\times[1,2]\) and get the same result. Then, using fubini's theorem,
    \[\int_Bx^2y^3=\int_{\overline A}u^5v=\int_1^{\sqrt2}u^5\dd u\int_1^{2}v\dd v=\frac{7}{4}\]
\end{document}