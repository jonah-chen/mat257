\documentclass{exam}
\title{MAT257 PSET 10---Question 1}
\author{Jonah Chen}
\date{}
\usepackage[utf8]{inputenc}
\usepackage[margin=0.5in]{geometry}

\usepackage{braket}
\usepackage{physoly}
\usepackage{currfile}
\usepackage{gensymb}
\usepackage{amssymb}
\usepackage{pgf,tikz,pgfplots}
\usepackage{mathrsfs}
\usepackage{textcomp}
\usepackage{parskip}
\setlength{\parindent}{0em}
\usetikzlibrary{arrows}
\numberwithin{equation}{section}
\pgfplotsset{compat=1.16}
\everymath{\displaystyle}
\newcommand{\R}{\mathbb{R}}

\begin{document}
    \sffamily
    \maketitle
    Let \(U=\{(r,\phi,\theta):0<r<a, 0<\phi<\pi,0<\theta<2\pi\}\) be open set and $g$ be the coordinate transformation \[g(r,\phi,\theta)=(r\cos\phi\cos\theta,r\cos\phi\sin\theta,r\sin\phi).\] Then, \(g(U)=\{(x,y,z):x^2+y^2+z^2<a^2, z>0\}=V\) and if \(f(x,y,z)=z\) then \(f(g(r,\phi,\theta))=r\sin\phi\). \(g'\) is also continuous on all of \(\R^3\). Using mathematica, it is possible to compute \(\det(g')=r^2\cos\phi.\) Thus, there are no $x\in U$ where \(\det(g')=0\). By the change of variables theorem,
    \begin{equation*}
        \int_{g(U)}z=\int_U(f\circ g)|\det(g')|=\int_Ur^3\sin\phi|\cos\phi|
    \end{equation*}
     Then, by fubini's theorem, we can integrate over the compact set \(\overline U\)
    \begin{equation*}
        \int_{\overline U}r^3\sin\phi|\cos\phi|=\int_0^ar^3\dd r\left(\int_0^\pi\sin\phi|\cos\phi|\dd\phi\left(\int_0^{2\pi}\dd\theta\right)\right)=\frac{\pi a^4}{2}.
    \end{equation*}
    As \(U\) has content zero boundary, so \[\int_U(f\circ g)|\det(g')|=\int_{\overline U}(f\circ g)|\det(g')|=\frac{\pi a^4}{2}.\]
\end{document}