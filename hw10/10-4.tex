\documentclass{exam}
\title{MAT257 PSET 10---Question 4}
\author{Jonah Chen}
\date{}
\usepackage[utf8]{inputenc}
\usepackage[margin=0.5in]{geometry}

\usepackage{braket}
\usepackage{physoly}
\usepackage{currfile}
\usepackage{gensymb}
\usepackage{amssymb}
\usepackage{pgf,tikz,pgfplots}
\usepackage{mathrsfs}
\usepackage{textcomp}
\usepackage{parskip}
\setlength{\parindent}{0em}
\usetikzlibrary{arrows}
\numberwithin{equation}{section}
\pgfplotsset{compat=1.16}
\everymath{\displaystyle}
\newcommand{\R}{\mathbb{R}}

\begin{document}
    \sffamily
    \maketitle
    Consider the linear transformation \(G=\begin{pmatrix}
    1 & 0 & -1 \\
    2 & 1 & 1 \\
    3 & 2 & 1
    \end{pmatrix}\). Then, \(G([0,1]^3)=T\). As \(G\) is a linear transformation, \(G\) is \(C^1\) and \(G'=G\) so \(\det(G')=\det(G)=-2,\) which is never zero. As \(T\)'s boundary is content zero, we can integrate over \((0,1)^3\) to get the same answer.
    
    Note that \(f(x,y,z)=x+2y-z\) is the linear transformation from \(F:\R^3\to\R\) where \(F=\begin{pmatrix}
        1 & 2 & -1
    \end{pmatrix}\). Then, \(F\circ G=\begin{pmatrix}2 & 0 & 0\end{pmatrix}\).
    
    Therefore, by change of variables theorem,
    \[\int_Tx+2y-z=\int_{(0,1)^3}(F\circ G)|\det(G')|=2\int_{(0,1)^3}\begin{pmatrix}
        2 & 0 & 0 \\
    \end{pmatrix}\]

    By fubini's theorem
    \[2\int_{[0,1]^3}\begin{pmatrix}
        2 & 0 & 0
    \end{pmatrix}=2\int_0^1\dd z\int_0^1\dd y\int_0^1 2x\dd x=2\]
\end{document}