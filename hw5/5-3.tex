\documentclass{exam}
\title{MAT257 PSET 5---Question 3}
\author{Jonah Chen}

\usepackage[utf8]{inputenc}
\usepackage[margin=0.5in]{geometry}

\usepackage{braket}
\usepackage{physoly}
\usepackage{currfile}
\usepackage{gensymb}
\usepackage{amssymb}
\usepackage{pgf,tikz,pgfplots}
\usepackage{mathrsfs}
\usepackage{textcomp}
\usepackage{parskip}
\setlength{\parindent}{0em}
\usetikzlibrary{arrows}
\numberwithin{equation}{section}
\pgfplotsset{compat=1.16}
\everymath{\displaystyle}
\newcommand{\R}{\mathbb{R}}
\begin{document}
    \sffamily
    \maketitle
    \begin{enumerate}[label=(\alph*)]
        \item Suppose $f$ is not 1-1. Then, there exists two distinct numbers $a,b\in\R$ such that $f(a)=f(b)$. WLOG suppose $a<b$. As $f$ is differentiable on $\R$, $f$ is continuous on $[a,b]$ and $f$ is differentiable on $(a,b)$. Therefore, by special case of mean value theorem (Rolle's), $\exists c\in (a,b)$ where $f'(c)=0$, which contridicts $f'(a)\neq0\forall a\in\R$.
        
        \item Clearly, $f$ is not 1-1 as $f(0,0)=f(0,2\pi)=(0, 0)$.
        
        We can find $f'(x,y)$ by using the partial derivatives of $f$.
        \begin{align*}
            f'(x,y)=\begin{pmatrix}
                \partial_x f_1 & \partial_y f_1\\
                \partial_x f_2 & \partial_y f_2
            \end{pmatrix}=\begin{pmatrix}
                e^x\cos y&-e^x\sin y\\
                e^x\sin y&e^x\cos y
            \end{pmatrix}&=e^x\begin{pmatrix}
                \cos y&-\sin y\\
                \sin y&\cos y
            \end{pmatrix}
        \end{align*}
        This is just a nonzero quantity $e^x$ multipled by a rotation matrix by angle $y$. Thus, the inverse should be 
        \begin{align*}
            (f')^{-1}(x,y)=e^{-y}\begin{pmatrix}
                \cos y&\sin y\\-\sin y&\cos y
            \end{pmatrix}
        \end{align*}
        We can verify that this is true for all $(x,y)\in\R^2$ by verifying that $(f')\circ(f')^{-1}=I$ for all $x$ and $y$.  Thus,
        \begin{align*}
            (f')(f')^{-1}(x,y)&=e^x\begin{pmatrix}
                \cos y&-\sin y\\
                \sin y&\cos y
            \end{pmatrix}e^{-y}\begin{pmatrix}
                \cos y&\sin y\\-\sin y&\cos y
            \end{pmatrix}\\
            &=\begin{pmatrix}
                \cos^2y+\sin^2y&-\sin y\cos y+\sin y\cos y\\
                -\sin y\cos y+\sin x\cos y&\cos^2y+\sin^2y
            \end{pmatrix}\\
            &=\begin{pmatrix}
                1&0\\0&1
            \end{pmatrix}=I
        \end{align*}
    \end{enumerate}
\end{document}