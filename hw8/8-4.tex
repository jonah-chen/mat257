\documentclass{exam}
\title{MAT257 PSET 8---Question 4}
\author{Jonah Chen}
\date{}
\usepackage[utf8]{inputenc}
\usepackage[margin=0.5in]{geometry}

\usepackage{braket}
\usepackage{physoly}
\usepackage{currfile}
\usepackage{gensymb}
\usepackage{amssymb}
\usepackage{pgf,tikz,pgfplots}
\usepackage{mathrsfs}
\usepackage{textcomp}
\usepackage{parskip}
\setlength{\parindent}{0em}
\usetikzlibrary{arrows}
\numberwithin{equation}{section}
\pgfplotsset{compat=1.16}
\everymath{\displaystyle}
\newcommand{\R}{\mathbb{R}}
\newcommand{\Comment}[1]{\:\:\:\:\:\:\:\:\:\:\:\:\:\:\:\:\:\:\text{#1}}
\begin{document}
    \sffamily
    \maketitle
    As $g_1, g_2$ are both $C^1$ functions,
    \begin{align*}
        D_1f(x,y)&=g_1(x,0)+D_1\int_0^yg_2(x,t)\dd t\Comment{  by FTC on }g_1\\
        &=g_1(x,0)+\int_0^yD_1g_2(x,t)\dd t\Comment{  by Leibnitz' rule}\\
        &=g_1(x,0)+\int_0^yD_2g_1(x,t)\dd t\Comment{  because } D_1g_2=D_2g_1\\
        &=g_1(x,0)+g_1(x,y)-g_1(x,0)\Comment{by FTC on }g_2\\
        &=g_1(x,y)
    \end{align*}

    As $\int_0^xg_1(t,0)\dd t$ is independent of the second variable $y$, by fundamental theorem of calculus,
    \begin{align*}
        D_2f(x,y)=D_2\int_0^yg_2(x,t)\dd t=g_2(x,y)
    \end{align*}

\end{document}