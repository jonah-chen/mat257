\documentclass{exam}
\title{MAT257 PSET 6---Question 4}
\author{Jonah Chen}

\usepackage[utf8]{inputenc}
\usepackage[margin=0.5in]{geometry}

\usepackage{braket}
\usepackage{physoly}
\usepackage{currfile}
\usepackage{gensymb}
\usepackage{amssymb}
\usepackage{pgf,tikz,pgfplots}
\usepackage{mathrsfs}
\usepackage{textcomp}
\usepackage{parskip}
\setlength{\parindent}{0em}
\usetikzlibrary{arrows}
\numberwithin{equation}{section}
\pgfplotsset{compat=1.16}
\everymath{\displaystyle}
\newcommand{\R}{\mathbb{R}}
\begin{document}
    \sffamily
    \maketitle
    For any vector $x\in\R^k\times\R^n$ define $x_X:=\begin{pmatrix}
        x_1&\dots&x_k
    \end{pmatrix}\in\R^k$ and $x_Y=\begin{pmatrix}
        x_{k+1}&\dots&x_{k+n}
    \end{pmatrix}\in\R^n$. 
    
    Then, $f(a)=f(a_X,a_Y)=0$. WLOG, assume $\partialderivative{f}{a_Y}$ has rank $n$ hence invertible (the choice of at least one set of $n$ coordinates for $a_Y$ must make this true as $f'(a)$ is rank $n$). 
    
    Define $R:=\det\left(\partialderivative{f}{a_Y}\right)$ which is nonzero as $\partialderivative{f}{a_Y}$ is full rank. Define $g(x_X,x_Y):=(x_X,f(x_X,x_Y))$. 
    
    Then, $Dg(a)=\begin{pmatrix}
        I_{k\times k} & 0\\
        \partialderivative{f}{a_X}&\partialderivative{f}{a_Y}
    \end{pmatrix}$ and hence $\det (Dg(a))=R\neq0$.

    As $g$ is also $C^1$ on $\R^k\times\R^n$ because $f$ is $C^1$ on $\R^k\times\R^n$, and $\R^k\times\R^n$ contains $a$, $g$ satisfies the hypotheses of the inverse function theorem.
    
    Hence, $\exists$ an open set $V\ni a$ and an open set $U\ni g(a)=(a_X,0)$ such that $g:V\to U$ has a continuous inverse $g^{-1}:U\to V$. 
    
    As $U$ is an open set, $\exists$ an open ball $B\subset U$ of radius $r>0$ centered at $(a_X,0)$.

    For any $c\in\R^n$ where $|c|<r,\, |(a_X,c)-(a_X,0)|<r$ so $(a_X,c)\in B\subset U\implies g^{-1}(a_X,c)$ exists. 
    
    Thus, $g(g^{-1}(a_X,c))=(a_X,c)\implies f(g^{-1}(a_X,c))=c$ by the definition of $g$ so the solution to the equation $f(x)=c$ is $$x=g^{-1}(a_X,c)$$
    for any $c$ sufficiently close to $0$.
\end{document}