\documentclass{exam}
\title{MAT257 PSET 6---Question 3}
\author{Jonah Chen}

\usepackage[utf8]{inputenc}
\usepackage[margin=0.5in]{geometry}

\usepackage{braket}
\usepackage{physoly}
\usepackage{currfile}
\usepackage{gensymb}
\usepackage{amssymb}
\usepackage{pgf,tikz,pgfplots}
\usepackage{mathrsfs}
\usepackage{textcomp}
\usepackage{parskip}
\setlength{\parindent}{0em}
\usetikzlibrary{arrows}
\numberwithin{equation}{section}
\pgfplotsset{compat=1.16}
\everymath{\displaystyle}
\newcommand{\R}{\mathbb{R}}
\begin{document}
    \sffamily
    \maketitle
    WLOG, assume $\partialderivative{(f,g)}{(y,z)}$ has rank 2 at $p_0$, hence it is invertible (as $\partialderivative{(f,g)}{(x,y,z)}$ is rank 2 at $p_0$, so the statement must true for at least one of $(x,y), (x,z), (y,z)$ and the rest of the proof is the same no matter which choice was made).
    
    Let $F=(f,g)$. $f$ and $g$ are $C^1$ so $F$ is $C^1$, the implicit function theorem guarentees the existance of $h(x)$ such that $F(x,h_1(x),h_2(x))=0$ in an open neighborhood about $p_0$.

    Thus locally, the curve described by $f(x,y,z)=0\land g(x,y,z)=0$ can be parameterized with $x=x, y=h_1(x), z=h_2(x)$.


\end{document}