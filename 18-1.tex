\documentclass{exam}
\title{MAT257 PSET 18---Question 1}
\author{Jonah Chen}
\date{}
\usepackage[utf8]{inputenc}
\usepackage[margin=0.5in]{geometry}

\usepackage{braket}
\usepackage{physoly}
\usepackage{currfile}
\usepackage{gensymb}
\usepackage{amssymb}
\usepackage{pgf,tikz,pgfplots}
\usepackage{mathrsfs}
\usepackage{textcomp}
\usepackage{parskip}
\usepackage{bbm}
\setlength{\parindent}{0em}
\usetikzlibrary{arrows}
\pgfplotsset{compat=1.16}
\everymath{\displaystyle}
\newcommand{\R}{\mathbb{R}}

\begin{document}
    \sffamily
    \maketitle
    \begin{enumerate}[label=\alph*)]
        \item Consider the 1-manifold-with-boundary $M=\R_+$ with the usual orientation and an atlas with one coordinate chart (the identity map from $\R_+\to\R_+$), and the $0$-form $\omega=x$, then $\dd\omega=\dd x$. We know $\partial M=-(0),$ so
        \begin{equation}
            \int_{\partial M}\omega=\int_{-(0)}x=0
        \end{equation}
        However, the other side
        \begin{equation}
            \int_M\dd\omega=\int_M\dd x=\int_{\R^+}1
        \end{equation}
        This integral does not exist, hence it is not equal to 1 and Stokes' theorem does not hold for non-compact manifolds. (which is well known as it's a single variable integral, but this can also be shown by using a partition of unity and showing the series does not converge)
        \begin{proposition}
            Let $\omega$ be a $k$-form supported on a compact set $A\subset M$. Then,
            \begin{equation}
                \int_{M}\omega=\int_A\omega
            \end{equation}
            \begin{proof}
                Consider
            \end{proof}
        \end{proposition}
        Then, if $\omega$ has compact support, $\dd\omega$ has the same support. Then,
        \begin{equation}
            \int_M\dd\omega=\int_{\mathrm{supp}(\omega)}\dd\omega=\int_{\partial\mathrm{supp}(\omega)}\omega
        \end{equation} 

        \item Given an exact $k$-form $\omega$ on a $k$-manifold-without-boundary $M$, by definition, $\omega=\dd\eta$ for some $k-1$-form $\eta$. If $M$ is compact and oriented, Stokes' theorem holds.
        \begin{equation}
            \int_M\omega=\int_M\dd\eta=\int_{\partial M}\eta=\int_{\emptyset}\eta=0
        \end{equation}

        If $M$ is not compact however, consider $M=\R$, and a closed form $\omega=\frac{e^{-x^2/2}}{\sqrt{2\pi}}\dd x=\dd\eta$ where $\eta=\frac{1}{2}\erf(x/\sqrt2)$.

        However, the integral
        \begin{equation}
            \int_M\omega=1
        \end{equation}
    \end{enumerate}
\end{document}