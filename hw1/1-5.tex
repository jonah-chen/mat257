\documentclass[a4paper]{article}
\title{MAT257 PSET 1---Question 5}
\author{Jonah Chen}

\usepackage[utf8]{inputenc}
\usepackage[margin=0.5in]{geometry}

\usepackage{braket}
\usepackage{physoly}
\usepackage{currfile}
\usepackage{gensymb}
\usepackage{amssymb}
\usepackage{pgf,tikz,pgfplots}
\usepackage{mathrsfs}
\usepackage{textcomp}
\usepackage{parskip}
\setlength{\parindent}{0em}
\usetikzlibrary{arrows}
\numberwithin{equation}{section}
\pgfplotsset{compat=1.16}

\newcommand{\R}{\mathbb{R}}
\newcommand{\bd}{\mathrm{bd }}
\begin{document}
\sffamily
As $A$ is the union of open intervals, $A$ is an open set. Thus, $\bd A\cap A=\emptyset$.

Next, show $\bd A\subset[0,1]$.
\begin{itemize}
    \item If $x>1$, consider the open rectangle $R=(1,x+1)\ni x$. Clearly, $A\cap R=A\cap (1,x+1)=\emptyset$. Hence, $x\notin\bd A$. 
    \item If $x<0$, consider the open rectangle $R=(x-1,0)\ni x$. Clearly, $A\cap R=A\cap (x-1,0)=\emptyset$. Hence, $x\notin\bd A$. 
\end{itemize}
Thus, $x\in\bd A\implies 0\leq x\leq 1\implies\bd A\subset[0,1]$. 

Consider $x\in \bd A$. $x\notin A$ because $\bd A\cap A=\emptyset$ and $x\in[0,1]$ because $\bd A\subset[0,1]$. Hence, $[0,1]\setminus A\supset\bd A$.

Next, show $[0,1]\setminus A\subset\bd A$. For any $x\in[0,1]:x\notin A$, consider every open rectangle $R=(a,b)\ni x$ where $a<b$.

Due to the density of $\mathbb Q$ over $\R$, there exists a rational in every open interval. 
\begin{itemize}
    \item If $0\leq a<b\leq 1$, $\exists y\in\mathbb Q:y\in(a,b)\subset(0,1)$.
    \item If $a<0<b<1$, $(a,b)=(a,0)\cup\{0\}\cup(0,b)$. Hence, $\exists y\in\mathbb Q:y\in(0,b)\subset(0,1)$.
    \item If $0<a<1<b$, $(a,b)=(a,1)\cup\{1\}\cup(1,b)$. Hence, $\exists y\in\mathbb Q:y\in(a,1)\subset(0,1)$.
    \item If $a<0<1<b$, $(a,b)=(a,0)\cup\{0\}\cup(0,1)\cup\{1\}\cup(1,b)$. Hence, $\exists y\in\mathbb Q:y\in(0,1)$.
\end{itemize}
Since all rationals in $(0,1)$ are contained in $A$, $y\in A$. Hence, $x\in R\cap A^C$ and $y\in R\cap A$ for every open rectangle $R\ni x$. 

Hence, $x\in [0,1]\setminus A\implies x\in\bd A$ and $[0,1]\setminus A\subset \bd A$.

As it was shown that $[0,1]\setminus A\subset\bd A$ and $[0,1]\setminus A\supset\bd A$, then $[0,1]\setminus A=\bd A$ as desired.
\end{document}