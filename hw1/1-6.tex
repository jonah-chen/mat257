\documentclass[a4paper]{article}
\title{MAT257 PSET 1---Question 6}
\author{Jonah Chen}

\usepackage[utf8]{inputenc}
\usepackage[margin=0.5in]{geometry}

\usepackage{braket}
\usepackage{physoly}
\usepackage{currfile}
\usepackage{gensymb}
\usepackage{amssymb}
\usepackage{pgf,tikz,pgfplots}
\usepackage{mathrsfs}
\usepackage{textcomp}
\usepackage{parskip}
\setlength{\parindent}{0em}
\usetikzlibrary{arrows}
\numberwithin{equation}{section}
\pgfplotsset{compat=1.16}

\newcommand{\R}{\mathbb{R}}
\newcommand{\bd}{\mathrm{bd }}
\begin{document}
\sffamily
\noindent Note that 
\begin{enumerate}
    \item $0\in A$ and $1\in A$ because they are rational. 
    \item $[0,1]\cap\mathbb Q\subset A$ as $A$ contains every rational number in $[0,1]$.
\end{enumerate}
Suppose $\exists x\in[0,1]\notin A$. Then, $x\in A^C$. As $A$ is closed, $A^C$ is open. 

\noindent Hence, $\exists$ open rectangle $R=(a,b)\subset A^C$. WLOG, assume $0\leq a<b\leq 1$. By the density of $\mathbb Q$, $\exists y\in R\cap\mathbb Q\subset [0,1]\cap\mathbb Q\subset A$. 

\noindent But also, $y\in R\subset A^C$, which contradicts $y\in A$; hence, there cannot exist $x\in[0,1]\notin A$ so $[0,1]\subset A$.  
\end{document}