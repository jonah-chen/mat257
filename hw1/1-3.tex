\documentclass[a4paper]{article}
\title{MAT257 PSET 1---Question 3}
\author{Jonah Chen}

\usepackage[utf8]{inputenc}
\usepackage[margin=0.5in]{geometry}

\usepackage{braket}
\usepackage{physoly}
\usepackage{currfile}
\usepackage{gensymb}
\usepackage{amssymb}
\usepackage{pgf,tikz,pgfplots}
\usepackage{mathrsfs}
\usepackage{textcomp}
\usepackage{parskip}
\setlength{\parindent}{0em}
\usetikzlibrary{arrows}
\numberwithin{equation}{section}
\pgfplotsset{compat=1.16}

\newcommand{\R}{\mathbb{R}}
\begin{document}
\sffamily
\noindent First, we show $T$ is linear. For any $x,y,z\in\R^n$ and $a,b\in\R$,
\begin{align*}
    [T(ax+by)]z=\langle ax+by,z\rangle=a\langle x,z\rangle+b\langle y,z\rangle=[aT(x)+bT(y)]z
\end{align*}
Next, by way of contradiction assume $\exists x,y\in\R^n, x\neq y:Tx=Ty$.
\begin{align*}
    \implies\forall z\in\R^n, \langle x,z\rangle=\langle y,z\rangle.
\end{align*}
Choose n vectors to be $z$. The standard basis vectors $e_1,\dots,e_n$.
\begin{align*}
    \langle x,e_i\rangle&=x_i, \langle y,e_i\rangle=y_i\\
    \implies \forall i&=1,\dots,n, x_i=y_i\\
    &\therefore x=y
\end{align*}
This is a contradiction thus $T$ is one-to-one. For linear transformations from $\R^n$ to $\R^n$, injective, surjective, and bijective are equivalent.
\end{document}