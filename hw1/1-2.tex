\documentclass[a4paper]{article}
\title{MAT257 PSET 1---Question 2}
\author{Jonah Chen}

\usepackage[utf8]{inputenc}
\usepackage[margin=0.5in]{geometry}

\usepackage{braket}
\usepackage{physoly}
\usepackage{currfile}
\usepackage{gensymb}
\usepackage{amssymb}
\usepackage{pgf,tikz,pgfplots}
\usepackage{mathrsfs}
\usepackage{textcomp}
\usepackage{parskip}
\setlength{\parindent}{0em}
\usetikzlibrary{arrows}
\numberwithin{equation}{section}
\pgfplotsset{compat=1.16}

\newcommand{\R}{\mathbb{R}}
\begin{document}
    \sffamily
    We can write $T$ in its matrix representation, consisting of the entries $T_{ij}$. Similarly, $h$ has its vector components $h_j$. 
    Note that \begin{align*}
        |h|^2=\sum_j (h_j)^2
    \end{align*}
    % \begin{lemma}
    %     Given non-negative $x_1,\dots,x_n$
    %     \begin{align*}
    %         \left(\sum_{i=1}^nx_i\right)^2\leq\sum_{i=1}^nx_i^2
    %     \end{align*}
    %     \begin{proof}
    %         Consider the partial sums $P_m:=\left(\sum_{i=1}^mx_i\right)^2, Q_m:=\sum_{i=1}^mx_i^2$.

    %         It is trivial that $P_1\leq Q_1$ since $x_1^2\leq x_1^2$. 

    %         For induction, assume $P_{m}\leq Q_m$. Then, we need to show
    %         \begin{align*}
    %             P_{m+1}&\leq Q_{m+1}\\
    %             (\sqrt{P_m}+x_{m+1})^2&\leq Q_m+x_{m+1}^2\\
    %             P_m+\sqrt{P_m}x_{m+1}+x_{m+1}^2&\leq Q_m+x_{m+1}^2\\
    %             P_m\leq P_m+\sqrt{P_m}x_{m+1}&\leq Q_m
    %         \end{align*} 

    %     \end{proof}
    % \end{lemma}
    Consider the quantity
    \begin{align*}
        M^2=\max_{i}\sum_{j=1}^n (T_{ij})^2
    \end{align*}
    Then, as matrix multiplication is
    \begin{align*}
        T(h)&=\sum_j T_{ij}h_j\\
        |T(h)|^2&=\sum_{i=1}^m\left((h_i)^2\sum_{j=1}^n(T_{ij})^2\right)\leq\sum_{i=1}^m(h_i)^2M^2=M^2|h|^2\\
        |T(h)|&\leq M|h|
    \end{align*}
\end{document}