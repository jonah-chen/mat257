\documentclass{exam}
\title{MAT257 PSET 7---Question 1}
\author{Jonah Chen}

\usepackage[utf8]{inputenc}
\usepackage[margin=0.5in]{geometry}

\usepackage{braket}
\usepackage{physoly}
\usepackage{currfile}
\usepackage{gensymb}
\usepackage{amssymb}
\usepackage{pgf,tikz,pgfplots}
\usepackage{mathrsfs}
\usepackage{textcomp}
\usepackage{parskip}
\setlength{\parindent}{0em}
\usetikzlibrary{arrows}
\numberwithin{equation}{section}
\pgfplotsset{compat=1.16}
\everymath{\displaystyle}
\newcommand{\R}{\mathbb{R}}
\begin{document}
    \sffamily
    \maketitle
    \begin{enumerate}[label=(\alph*)]
        \item For a subrectangle $S$ and $y\in S$, $m_S(f)\leq f(y)$ and $m_S(y)\leq g(y)$ so $m_S(f)+m_S(g)\leq f(y)+g(y)$. Since this is true for any $y\in S$, $m_S(f)+m_S(g)\leq m_S(f+g)$.
        
        Similarly, $M_S(f)\geq f(y)$ and $M_S(g)\geq g(y)$ so $M_S(f)+M_S(g)\geq M_S(f+g)$.

        For any partition $P$ of $A$,
        \begin{align*}
            L(f+g,P)=\sum_{S\in P} m_S(f+g) v(S)\geq\sum_{S\in P} m_S(f) v(S) + \sum_{S\in P} m_S(g) v(S)=L(f,P)+L(g,P)\\
            U(f+g,P)=\sum_{S\in P} M_S(f+g) v(S)\leq\sum_{S\in P} M_S(f) v(S) + \sum_{S\in P} M_S(g) v(S)=U(f,P)+U(g,P)\\
        \end{align*}

        \item To show $f+g$ is integrable on $A$, and given $\varepsilon>0$, we need to find a partition $P$ such that $U(f+g,P)-L(f+g,P)<\varepsilon$. 
        
        As $f$ and $g$ are integrable on $A$, there exists partitions of $A$, $P_f$ and $P_g$, such that $U(f,P_f)-L(f,P_f)<\varepsilon/2$ and $U(g,P_g)-L(g,P_g)<\varepsilon/2$.

        Let $P$ be the refinement of $P_f$ and $P_g$. Then,
        \begin{align*}
            U(f,P)-L(f,P)\leq U(f,P_f)-L(f,P_f)<\varepsilon/2\\
            U(g,P)-L(g,P)\leq U(g,P_g)-L(g,P_g)<\varepsilon/2
        \end{align*}
        For $f+g$, $L(f,P)+L(g,P)\leq L(f+g,P)\leq U(f+g,P)\leq U(f,P)+U(g,P)$. Hence, $U(f+g,P)-L(f+g,P)<\varepsilon$.
        
        For any $P_f$ and $P_g$ of $A$, let $P$ be the refinement of $P_f$ and $P_g$. Then,
        \begin{align*}
            L(f,P)\geq L(f,P_f)\\
            L(g,P)\geq L(g,P_g)\\
            U(f,P)\leq U(f,P_f)\\
            U(g,P)\leq U(g,P_g)
        \end{align*}
        So,
        \begin{align*}
            \sup_{P_f} L(f,P_f)+\sup_{P_g} L(g,P_g)=\sup_P\{L(f,P)+L(g,P)\}\\
            \inf_{P_f} U(f,P_f)+\inf_{P_g} U(g,P_g)=\inf_P\{U(f,P)+U(g,P)\}
        \end{align*}

        Then, $$\int_Af+\int_Ag=\sup_P\{L(f,P)+L(g,P)\}\leq\sup_PL(f+g,P)\leq\inf_PU(f+g,P)\leq\inf_P\{U(f,P)+U(g,P)\}=\int_Af+\int_Ag$$

        Thus, $\sup_PL(f+g,P)=\inf_PU(f+g,P)=\int_Af+g=\int_Af+\int_Ag$.

        \item Case $c=0:cf=0\implies\int_A0=0\int_Af=0$ as $f$ is integrable on $A$.
        
        Case $c>0$: For any partition $P$ of $A$,
        \begin{align*}
            L(cf,P)=\sum_{S\in P} m_S(cf) v(S)=c\sum_{S\in P}m_S(cf)v(S)=cL(f,P)\\
            U(cf,P)=\sum_{S\in P} M_S(cf) v(S)=c\sum_{S\in P}M_S(cf)v(S)=cU(f,P)
        \end{align*}
        Then, $\sup L(cf,P)=c\sup L(f,P)$ and $\inf U(cf,P)=c\inf U(f,P)$. As $f$ is integrable on $A$, 
        
        $$\sup L(f,P)=\inf U(f,P)=\int_Af\implies\sup L(cf,P)=\inf U(cf,P)=\int_Acf=c\int_Af$$
        
        Case $c<0$. Then, for any partition $P$ of $A$,
        \begin{align*}
            L(cf,P)=\sum_{S\in P} m_S(cf) v(S)=c\sum_{S\in P}M_S(f)v(S)=cU(f,P)\\
            U(cf,P)=\sum_{S\in P} M_S(cf) v(S)=c\sum_{S\in P}m_S(cf)v(S)=cL(f,P)
        \end{align*}
        
        Then, $\sup L(cf,P)=c\inf U(f,P)$ and $\inf U(cf,P)=c\sup L(f,P)$. As $f$ is integrable on $A$, 
        
        $$\sup L(f,P)=\inf U(f,P)=\int_Af\implies\sup L(cf,P)=\inf U(cf,P)=\int_Acf=c\int_Af$$

    \end{enumerate}
\end{document}