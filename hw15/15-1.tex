\documentclass{exam}
\title{MAT257 PSET 15---Question 1}
\author{Jonah Chen}
\date{}
\usepackage[utf8]{inputenc}
\usepackage[margin=0.5in]{geometry}

\usepackage{braket}
\usepackage{physoly}
\usepackage{currfile}
\usepackage{gensymb}
\usepackage{amssymb}
\usepackage{pgf,tikz,pgfplots}
\usepackage{mathrsfs}
\usepackage{textcomp}
\usepackage{parskip}
\usepackage{bbm}
\setlength{\parindent}{0em}
\usetikzlibrary{arrows}
\numberwithin{equation}{section}
\pgfplotsset{compat=1.16}
\everymath{\displaystyle}
\newcommand{\R}{\mathbb{R}}

\begin{document}
    \sffamily
    \maketitle
    \begin{enumerate}[label=\alph*)]
        \item We know that \[\dd f=\partialderivative{f}{x}\dd x+\partialderivative{f}{y}\dd y+\partialderivative{f}{z}\dd z.\] And \[\mathrm{grad}f=\partialderivative{f}{x}\partial_x+\partialderivative{f}{y}\partial_y+\partialderivative{f}{z}\partial_z\]
        Hence its component functions of \(\mathrm{grad}f\) are \(F_1=\partialderivative{f}{x},F_2=\partialderivative{f}{y},F_3=\partialderivative{f}{z}.\) Then,
        \[\omega_{\mathrm{grad}f}^1=\partialderivative{f}{x}\dd x+\partialderivative{f}{y}\dd y+\partialderivative{f}{z}\dd z=df,\]
        as desired. 

        We know that \[\mathrm{curl}f=\left(\partialderivative{F_3}{y}-\partialderivative{F_2}{z}\right)\partial_x+\left(\partialderivative{F_1}{z}-\partialderivative{F_3}{x}\right)\partial_y+\left(\partialderivative{F_2}{x}-\partialderivative{F_1}{y}\right)\partial_z.\]

        Hence, its component functions of \(G=\mathrm{curl}f\) are \(G_1=\partialderivative{F_3}{y}-\partialderivative{F_2}{z},G_2=\partialderivative{F_1}{z}-\partialderivative{F_3}{x},G_3=\partialderivative{F_2}{x}-\partialderivative{F_1}{y}.\)
        \begin{align*}
            \dd(\omega_F^1)&=\dd(F_1\dd x+F_2\dd y+F_3\dd z)\\
            &=\dd x\wedge\left(\partialderivative{F_1}{x}\dd x+\partialderivative{F_2}{x}\dd y+\partialderivative{F_3}{x}\dd z\right)+\dd y\wedge\left(\partialderivative{F_1}{y}\dd x+\partialderivative{F_2}{y}\dd y+\partialderivative{F_3}{y}\dd z\right)+\dd z\wedge\left(\partialderivative{F_1}{z}\dd x+\partialderivative{F_2}{z}\dd y+\partialderivative{F_3}{z}\dd z\right)\\
            &=\partialderivative{F_2}{x}\dd x\wedge\dd y+\partialderivative{F_3}{x}\dd x\wedge\dd z+\partialderivative{F_1}{y}\dd y\wedge\dd x+\partialderivative{F_3}{y}\dd y\wedge\dd z+\partialderivative{F_1}{z}\dd z\wedge\dd x+\partialderivative{F_2}{z}\dd z\wedge\dd y\\
            &=\left(\partialderivative{F_3}{y}-\partialderivative{F_2}{z}\right)\dd y\wedge\dd z+\left(\partialderivative{F_1}{z}-\partialderivative{F_3}{x}\right)\dd z\wedge\dd x+\left(\partialderivative{F_2}{x}-\partialderivative{F_1}{y}\right)\dd x\wedge\dd y\\
            &=G_1\dd y\wedge\dd z+G_2\dd z\wedge\dd x+G_3\dd x\wedge\dd y\\
            &=\omega_G^2=\omega_{\mathrm{curl}F}^2,
        \end{align*}
        as desired.

        We know that \[\mathrm{div} F=\partialderivative{F_1}{x}+\partialderivative{F_2}{y}+\partialderivative{F_3}{z}.\]
        And,
        \begin{align*}
            \dd(\omega_F^2)&=\dd(F_1\dd y\wedge\dd z+F_2\dd z\wedge\dd x+F_3\dd x\wedge\dd y)\\
            &=\partialderivative{F_1}{x}\dd x\wedge\dd y\wedge\dd z+\partialderivative{F_2}{y}\dd y\wedge\dd z\wedge\dd x+\partialderivative{F_3}{z}\dd z\wedge\dd x\wedge\dd y\\
            &=\left(\partialderivative{F_1}{x}+\partialderivative{F_2}{y}+\partialderivative{F_3}{z}\right)\dd x\wedge\dd y\wedge \dd z\\
            &=\mathrm{div}F\dd x\wedge\dd y\wedge\dd z,
        \end{align*}
        as desired.

        \item \(\omega_{\mathrm{curl\,grad}f}^2=\dd(\omega_{\mathrm{grad}f}^1)=\dd(\dd f)=\dd^2f=0\implies\) the component functions of \(\mathrm{curl\,grad}f\) are zero so \(\mathrm{curl\,grad}f=0.\)
        
        \((\mathrm{div\,curl}\,F)\,\dd x\wedge\dd y\wedge\dd z=\dd(\omega_{\mathrm{curl}\,F}^2)=\dd(\dd(\omega_F^1))=d^2(\omega_F^1)=0\implies\mathrm{div\,curl}\,F=0.\)

    \end{enumerate}
\end{document}