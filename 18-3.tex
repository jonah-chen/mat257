\documentclass{exam}
\title{MAT257 PSET 18---Question 3}
\author{Jonah Chen}
\date{}
\usepackage[utf8]{inputenc}
\usepackage[margin=0.5in]{geometry}

\usepackage{braket}
\usepackage{physoly}
\usepackage{currfile}
\usepackage{gensymb}
\usepackage{amssymb}
\usepackage{pgf,tikz,pgfplots}
\usepackage{mathrsfs}
\usepackage{textcomp}
\usepackage{parskip}
\usepackage{bbm}
\setlength{\parindent}{0em}
\usetikzlibrary{arrows}
\pgfplotsset{compat=1.16}
\everymath{\displaystyle}
\newcommand{\R}{\mathbb{R}}

\begin{document}
\sffamily
\maketitle
\begin{enumerate}[label=\alph*)]
    \item Consider $M=D^3_d\setminus\mathrm{int}\,D^3_c$. This is a compact manifold with boundary, with boundary $\partial M=S^2_d-S^2_c$, since $S^2_c$ has the opposite orientation. The integral \begin{equation*}
        \int_{M}\dd\omega=\int_{\partial M}\omega=\int_{S^2_d}\omega-\int_{S^2_c}\omega=(a+b/d)-(a+b/c)=b/d-b/c.
    \end{equation*}
    \item If $\omega$ is closed, $\dd\omega=0$. As $\omega$ is a $2$-form on $\R^3\setminus\{0\}$, we can use the result from part a, with $\int_M\dd\omega=0$ since $\omega$ is closed. Then we have $b/d=b/c$ for any $d>c>0$ hence $b=0$.
\end{enumerate}
\end{document}