\documentclass{exam}
\title{MAT257 PSET 13---Question 2}
\author{Jonah Chen}
\date{}
\usepackage[utf8]{inputenc}
\usepackage[margin=0.5in]{geometry}

\usepackage{braket}
\usepackage{physoly}
\usepackage{currfile}
\usepackage{gensymb}
\usepackage{amssymb}
\usepackage{pgf,tikz,pgfplots}
\usepackage{mathrsfs}
\usepackage{textcomp}
\usepackage{parskip}
\usepackage{bbm}
\setlength{\parindent}{0em}
\usetikzlibrary{arrows}
\numberwithin{equation}{section}
\pgfplotsset{compat=1.16}
\everymath{\displaystyle}
\newcommand{\R}{\mathbb{R}}

\begin{document}
    \sffamily
    \maketitle
    \begin{enumerate}[label=\alph*)]
        \item Using the standard basis for \(\R^2\), \(L_1\) can be represented as a matrix \(M_1=\begin{pmatrix}
            -1 & 0 \\
            0 & 1
        \end{pmatrix}\), and \(\det M_1=-1\) so \(L_1\) is orientation reversing.

        \item \(L_2\) is represented as a matrix \(M_2=\begin{pmatrix}
            0 & 1 \\
            1 & 0
        \end{pmatrix}\), and \(\det M_2=-1\) so \(L_2\) is orientation reversing.

        \item Rotations have determinent 1, so \(L_3\) is orientation preserving.
        \item Same as c, \(L_4\) is orientation preserving.
        \item Using \(a+bi\to (a,b)\), the complex conjugation map can be represented as a matrix \(M_5=\begin{pmatrix}
            1 & 0 \\
            0 & -1
        \end{pmatrix}\), and \(\det M_5=-1\) so \(L_5\) is orientation reversing.
        
        \item Using the standard basis for \(\R^3\), \(L_6\) can be represented as a matrix \(M_6=\begin{pmatrix}
            0 & 1 & 0 \\
            0 & 0 & 1 \\
            1 & 0 & 0
        \end{pmatrix}\), and \(\det M_6=1\) so \(L_6\) is orientation preserving.

        \item Using the standard basis for \(\R^n\), the matrix representing \(L_7\) is \(M_7=(-1)\mathbbm{1}_{n\times n}\), so \(\det M_7=(-1)^n\det\mathbbm 1=(-1)^n\), so \(L_6\) is orientation preserving if \(n\) is even, and orientation reversing if \(n\) is odd.
        \item Using the standard basis for \(\R^{m+n}\), the matrix describing \(L_8\) is \(M_8=\begin{pmatrix}
            0_{m\times n} & \mathbbm 1_{n\times n}\\
            \mathbbm 1_{m\times m} & 0_{n\times m}
        \end{pmatrix}\)

        We can define the matrix $P_j$ which swap the $j$-th and $j+1$-th rows of a matrix. This matrix has determinant $-1$. Then,
        \[
            (P_{n}P_{n-1}\dots P_1)(P_{n+1}P_n\dots P_2)\dots(P_{n+m-1}\dots P_{m})M_8=\mathbbm{1}. 
        \]

        There are a total of $nm$ of these $P_j$ matrices, so \((-1)^{nm}\det M_8=\det\mathbbm 1=1, \det M_8=(-1)^{nm}\), so \(L_8\) is orientation reversing if both \(m\) and \(n\) are odd, and orientation preserving otherwise.
    \end{enumerate}
\end{document}