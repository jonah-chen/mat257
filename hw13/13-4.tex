\documentclass{exam}
\title{MAT257 PSET 13---Question 4}
\author{Jonah Chen}
\date{}
\usepackage[utf8]{inputenc}
\usepackage[margin=0.5in]{geometry}

\usepackage{braket}
\usepackage{physoly}
\usepackage{currfile}
\usepackage{gensymb}
\usepackage{amssymb}
\usepackage{pgf,tikz,pgfplots}
\usepackage{mathrsfs}
\usepackage{textcomp}
\usepackage{parskip}
\usepackage{bbm}
\setlength{\parindent}{0em}
\usetikzlibrary{arrows}
\numberwithin{equation}{section}
\pgfplotsset{compat=1.16}
\everymath{\displaystyle}
\newcommand{\R}{\mathbb{R}}

\begin{document}
    \sffamily
    \maketitle
    For convinence, we will sum over repeated multi-indices in the same term.

    \begin{enumerate}[label=\alph*)]
        \item As the basis for \(V^*\) is already given, we can easily make a basis for \(\Lambda^k(V)\) as \(\{I\in\underline{n}_a^k: \varphi_I\}\), where $\underline{n}_a^k$ is the set of ascending multi-index of length $k$.
        
        We define \(I'\in\underline{n}_a^{n-k}, I':=\underline{n}\setminus I\), where $\underline{n}=\{1,\ldots,n\}$. 
        
        For \(I=\{i_1,\dots,i_k\},J=\{j_1,\dots,j_k\}\in\underline{n}_a^{k}\), define \(I+J=:\{i_1,\dots,i_k,j_1,\dots,j_k\}\).

        For \(I\in \underline{n}^k\), and let \(\tau_I\in S_n\) be the permutation where \(\tau_I\underline{n}=I\), define \((-1)^{I}=(-1)^{\tau_I}=(-1)^{\tau_I^{-1}}\).

        For any \(\lambda,\eta\in\Lambda^k(V)\) written in the basis \(\lambda=a_I\omega_I,\eta=b_J\omega_J\) and \(a,b\in\R\), we define the hodge star operator to be linear, so \[
        \star(a\lambda+b\eta)=a\star\lambda+b\star\eta.
        \]
        Thus, defining this operator on the basis vectors is sufficient. \[
            \star\omega_I=(-1)^{I+I'}\omega_{I'}.
        \]
        Note that for another basis vector, \(I\neq J\) i.e. \(I\cap J'\neq\emptyset\)\[
            \implies\omega_I\wedge\star\omega_J=(-1)^{J+J'}\omega_I\wedge\omega_{J'}=0
        \]
        Also,\[
            \omega_I\wedge\star\omega_I=(-1)^{I+I'}\omega_I\wedge\omega_{I'}=(-1)^{I+I'}\omega_{I+I'}=\omega_n
        \]
        So, we know that \(\omega_I\wedge\star\omega_J=\delta_{IJ}\omega_n\). Also, for \(a_I,b_J\in\R\) then due to the bilinearity property of the inner product, \(\langle a_I\omega_I,b_J\omega_J\rangle=a_Ib_J\delta_{IJ}\)

        Now, we show that the hodge star operator satisfies
        \begin{align*}
            \lambda\wedge(\star\eta)&=a_I\omega_I\wedge b_J\omega_J\\
            &=a_Ib_J(\omega_I\wedge\omega_J)\\
            &=a_Ib_J\delta_{IJ}\omega_n\\
            &=\langle\lambda,\eta\rangle\omega_n.
        \end{align*}
        
        As the operator is defined as linear, and both \(\Lambda^{k}(V)\) and \(\Lambda^{n-k}(V)\) have the same dimension because \(\begin{pmatrix}
            n\\k
        \end{pmatrix}=\begin{pmatrix}
            n\\n-k
        \end{pmatrix}\).
        Using the basis vectors \(\{\omega_I\}_{I\in\underline{n}_a^k}\) for \(\Lambda^k(V)\) and $\{\star\omega_I\}_{I\in \underline{n}_a^k}$ for \(\Lambda^{n-k}(V)\). Then the matrix representing the hodge star operator is the identity matrix, which means it is invertiable. Hence, the hodge star operator is invertible.

        \item Using the previous definition, for $n=3,k=1$, \(\star\omega_1=\omega_{23},\star\omega_2=-\omega_{13},\star\omega_3=\omega_{12}\). 
        
        For $n=4,k=2$, \(\star\omega_{12}=\omega_{34},\star\omega_{13}=-\omega_{24},\star\omega_{14}=\omega_{23},\star\omega_{23}=\omega_{14},\star\omega_{24}=-\omega_{13},\star\omega_{34}=\omega_{12}\).

        \item Given some alternating tensor \(\lambda\in\Lambda^k(V)\) and \(\lambda=a_I\omega_I\)\begin{align*}
            \star(\star\lambda)&=a_I\star(\star\omega_I)\\
            &=(-1)^{I+I'}a_I\star(\omega_{I'})\\
            &=(-1)^{I'+I}(-1)^{I+I'}a_I\omega_I\\
            &=(-1)^{I'+I}(-1)^{I+I'}\lambda.
        \end{align*}

        Define the permutation \(\tau_j\in S_n\) such that \(\tau_j\{a_1,\dots,a_n\}=\{a_1,\dots,a_{j-1},a_{j+1},a_j,\dots,a_n\}\). It is obvious that \((-1)^{\tau_j}=-1\).

        We can calculate the sign of $I'+I$ in terms of the sign of $I+I'$. As $I$ has length $k$ and $I'$ has length $n-k$, for element number $k+1,\dots,k+n$, making $k$ swaps between consecutive elements to position $1,\dots,k$ will convert $I+I'$ to $I'+I$. Performing the inverse operation require the same number of swaps.

        This is done by the permutation \((\tau_k\tau_{k-1}\dots\tau_1)(\tau_{k+1}\tau_k\dots\tau_2)\dots(\tau_{n-1}\dots\tau_{n-k})\).
        The number of elements to swap is $n-k$, and $k$ swaps per element hence $k(n-k)$ swaps are required to swap $I+I'$ to $I'+I$. Thus,
        \begin{align*}
            \star(\star\lambda)=(-1)^{I'+I}(-1)^{I+I'}\lambda=((-1)^{I'+I})^2(-1)^{k(n-k)}(\lambda)=(-1)^{k(n-k)}(\lambda).
        \end{align*}
        Hence, \(\star\circ\star=(-1)^{k(n-k)}\mathbbm 1\).
    \end{enumerate}
\end{document}