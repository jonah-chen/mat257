\documentclass{exam}
\title{MAT257 PSET 13---Question 3}
\author{Jonah Chen}
\date{}
\usepackage[utf8]{inputenc}
\usepackage[margin=0.5in]{geometry}

\usepackage{braket}
\usepackage{physoly}
\usepackage{currfile}
\usepackage{gensymb}
\usepackage{amssymb}
\usepackage{pgf,tikz,pgfplots}
\usepackage{mathrsfs}
\usepackage{textcomp}
\usepackage{parskip}
\usepackage{bbm}
\setlength{\parindent}{0em}
\usetikzlibrary{arrows}
\numberwithin{equation}{section}
\pgfplotsset{compat=1.16}
\everymath{\displaystyle}
\newcommand{\R}{\mathbb{R}}

\begin{document}
    \sffamily
    \maketitle
    For \(\omega\in\Lambda^{n-k}(V)\) and \(\lambda\in\Lambda^k(V)\). Define the function \(\psi_k:\Lambda^{n-k}(V)\to(\Lambda^k(V))^*\) as
    \[
        \psi_k(\omega)(\lambda) = \chi(\omega\wedge\lambda)
    \]
    We claim \(\psi_k\) is an isomorphism. Given any \(\omega,\xi\in\Lambda^{n-k}(V)\) and \(a,b\in\R\),
    \begin{align*}
        \psi_k(a\omega+b\xi)(\lambda)
        &=\chi((a\omega+b\xi)\wedge\lambda)\\
        &=\chi(a\omega\wedge\lambda+b\xi\wedge\lambda)\\
        &=a\chi(\omega\wedge\lambda)+b\chi(\xi\wedge\lambda)\\
        &=a\psi_k(\omega)+b\psi_k(\xi)\\
    \end{align*}
    Hence, \(\psi_k\) is linear. As \(\psi_k\) is a linear map between \(\begin{pmatrix}
        n\\k
    \end{pmatrix}\) dimensional vector spaces, we need to show that \(\psi_k\) is 1-to-1, which is equivalent to showing the null space contains only the zero vector. 
    
    By way of contradiction, suppose \(\exists\,\omega\neq0\in\Lambda^{n-k}(V): \psi_k(\omega)=0\). Then, for any \(\lambda\in\Lambda^{k}(V)\), $\psi_k(\omega)(\lambda)=\chi(\omega\wedge\lambda)=0$. As \(\chi\) is an isomorphism, \(\omega\wedge\lambda=0\) so for any $n$ vectors $v_1,\dots,v_n\in V$,
    \[
        \omega\wedge\lambda(v_1,\dots,v_n)=\frac{1}{n!(n-k)!}\sum\omega(v_1,\dots,v_k)\lambda(v_{k+1},\dots,v_n)
    \]
    
    (I can't really figure the rest out)

\end{document}