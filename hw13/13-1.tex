\documentclass{exam}
\title{MAT257 PSET 13---Question 1}
\author{Jonah Chen}
\date{}
\usepackage[utf8]{inputenc}
\usepackage[margin=0.5in]{geometry}

\usepackage{braket}
\usepackage{physoly}
\usepackage{currfile}
\usepackage{gensymb}
\usepackage{amssymb}
\usepackage{pgf,tikz,pgfplots}
\usepackage{mathrsfs}
\usepackage{textcomp}
\usepackage{parskip}
\usepackage{bbm}
\setlength{\parindent}{0em}
\usetikzlibrary{arrows}
\numberwithin{equation}{section}
\pgfplotsset{compat=1.16}
\everymath{\displaystyle}
\newcommand{\R}{\mathbb{R}}

\begin{document}
    \sffamily
    \maketitle
    For convinence, we will sum over repeated indices in the same term from 1 to 3. Let \(\varepsilon_{ijk}\) be the levi-civita tensor.

    Given the standard basis for \(\R^3, \{e_1,e_2,e_3\}\) and its dual basis \(\{\varphi_1,\varphi_2,\varphi_3\}\), we can identify a basis for \(\Lambda^1(\R^3)\): \(\{\varphi_1,\varphi_2,\varphi_3\}\) and for \(\Lambda^2(\R^3)\): \(\{k=1,2,3:\Phi_k:=\frac{1}{2}\varepsilon_{ijk}(\varphi_i\wedge\varphi_j)\}=\{\varphi_2\wedge\varphi_3,\varphi_3\wedge\varphi_1,\varphi_1\wedge\varphi_2\}\).

    Using these bases, the wedge product \(\wedge:\Lambda^1(\R^3)\times\Lambda^1(\R^3)\to\Lambda^2(\R^3)\) is the cross product \(P:\R^3\times\R^3\to\R^3, P(x,y)=x\times y\).

    Let \(\omega,\eta\in\Lambda^1(\R^3)\),
    \[
        \omega\wedge\eta = (a_i\varphi_i)\wedge(b_j\varphi_j) = a_ib_j(\varphi_i\wedge\varphi_j) = \varepsilon_{ijk}a_ib_j\Phi_k
    \]
    Similarly, for \(x=a_ie_i,y=b_je_j\in\R^3\), their cross product is 
    \[
        x\times y=(a_ie_i)\times (b_je_j) = \varepsilon_{ijk}a_ib_je_k
    \]

    As \(\{\varphi_i\}\) are the basis for \(\Lambda^1(\R^3)\) and \(\{\Phi_i\}\) are the basis for \(\Lambda^2(\R^3)\), these forms are equivalent.
\end{document}