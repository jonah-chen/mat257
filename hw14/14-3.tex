\documentclass{exam}
\title{MAT257 PSET 14---Question 3}
\author{Jonah Chen}
\date{}
\usepackage[utf8]{inputenc}
\usepackage[margin=0.5in]{geometry}

\usepackage{braket}
\usepackage{physoly}
\usepackage{currfile}
\usepackage{gensymb}
\usepackage{amssymb}
\usepackage{pgf,tikz,pgfplots}
\usepackage{mathrsfs}
\usepackage{textcomp}
\usepackage{parskip}
\usepackage{bbm}
\setlength{\parindent}{0em}
\usetikzlibrary{arrows}
\numberwithin{equation}{section}
\pgfplotsset{compat=1.16}
\everymath{\displaystyle}
\newcommand{\R}{\mathbb{R}}

\begin{document}
    \sffamily
    \maketitle
    As \(\gamma:[0,1]\to\R^2\) is differentiable, define coordinate functions \(\gamma_1,\gamma_2:[0,1]\to\R\) that are differentiable so that \[\gamma(t)=\gamma_1(t)e_1+\gamma_2(t)e_2.\] 

    As \(|\gamma(t)|=1\),\(f(t)=:\langle\gamma(t),\gamma(t)\rangle=(\gamma_1(t))^2+(\gamma_2(t))^2=1.\) Hence,\[
        f'(t)=2\gamma_1(t)\gamma_1'(t)+2\gamma_2(t)\gamma_2'(t)=2\langle\gamma'(t),\gamma(t)\rangle=0\implies\langle\gamma'(t),\gamma(t)\rangle=0\]
    The tangent vector to the curve \(\gamma(t)\) is \(\gamma'(t)_{\gamma(t)},\)
    \[\langle\gamma'(t)_{\gamma(t)},\gamma(t)_{\gamma(t)}\rangle=\langle\gamma'(t),\gamma(t)\rangle=0\] 
    so for any \(t\), the tangent vector to the curve \(\gamma'(t)_{\gamma(t)}\) is perpendicular to the \(\gamma(t)_{\gamma(t)}\).
\end{document}