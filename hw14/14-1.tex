\documentclass{exam}
\title{MAT257 PSET 14---Question 1}
\author{Jonah Chen}
\date{}
\usepackage[utf8]{inputenc}
\usepackage[margin=0.5in]{geometry}

\usepackage{braket}
\usepackage{physoly}
\usepackage{currfile}
\usepackage{gensymb}
\usepackage{amssymb}
\usepackage{pgf,tikz,pgfplots}
\usepackage{mathrsfs}
\usepackage{textcomp}
\usepackage{parskip}
\usepackage{bbm}
\setlength{\parindent}{0em}
\usetikzlibrary{arrows}
\numberwithin{equation}{section}
\pgfplotsset{compat=1.16}
\everymath{\displaystyle}
\newcommand{\R}{\mathbb{R}}

\begin{document}
    \sffamily
    \maketitle
    We know that if \(f:\R^n\to\R\) is a 0-form, \(\dd f=\sum_i\frac{\partial f}{\partial x^i}\dd x_i\). Since \(f,g,fg\) are 0-forms,
    \begin{align*}
        \dd(fg)&=\sum_i\frac{\partial(fg)}{\partial x^i} \dd x_i\\
        &=\sum_i\frac{\partial f}{\partial x^i}g+f\frac{\partial g}{\partial x^i} \dd x_i\\
        &=g\sum_i\frac{\partial f}{\partial x^i} \dd x_i+f\sum_i\frac{\partial g}{\partial x^i} \dd x_i\\
        &=f\dd g + g\dd f
    \end{align*}
\end{document}