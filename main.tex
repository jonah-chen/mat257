\documentclass{exam}
\title{MAT257 PSET 15---Question 5}
\author{Jonah Chen}
\date{}
\usepackage[utf8]{inputenc}
\usepackage[margin=0.5in]{geometry}

\usepackage{braket}
\usepackage{physoly}
\usepackage{currfile}
\usepackage{gensymb}
\usepackage{amssymb}
\usepackage{pgf,tikz,pgfplots}
\usepackage{mathrsfs}
\usepackage{textcomp}
\usepackage{parskip}
\usepackage{bbm}
\setlength{\parindent}{0em}
\usetikzlibrary{arrows}
\pgfplotsset{compat=1.16}
\everymath{\displaystyle}
\newcommand{\R}{\mathbb{R}}

\begin{document}
    \sffamily
    \maketitle

    \begin{enumerate}[label=\alph*)]
        \item The definition of the exterior derivative is that \(\dd\omega=:\sum_{i=1}^n\dd x_i\wedge\partialderivative{\omega}{x_i}.\)

        For the given \(\omega,\)
        \begin{align*}
            \dd\omega=\sum_{j=1}^n\sum_{i=1}^n(-1)^{i-1}\partial_j\left(\frac{x_i}{|x|^p}\right)\dd x_j\wedge\dd x_1\wedge\dots\hat{\dd x_i}\wedge\dots\dd x_n
        \end{align*}
        We notice that unless \(i=j\), the term is zero as there would be two copies of \(\dd x_i\) in the alternating wedge product. So we have
        \begin{align*}
            \dd\omega=\sum_{i=1}^n(-1)^{i-1}\partial_i\left(\frac{x_i}{|x|^p}\right)\dd x_i\wedge\dd x_1\wedge\dots\hat{\dd x_i}\wedge\dots\dd x_n
        \end{align*}
        It would take \(i-1\) swaps to move the first term to the correct place for the sequence to be ascending, which introduces a factor of \((-1)^{i-1}\).
        \begin{align*}
            \dd\omega=\dd x_1\wedge\dots\wedge\dd x_n\sum_{i=1}^n\partial_i\left(\frac{x_i}{|x|^p}\right)
        \end{align*}
        The partial derivative can be computed with quotient rule to be \(\frac{1-px_i^2/|x|^2}{|x|^p},\) thus,
        \begin{align*}
            \dd\omega&=\frac{1}{|x|^p}\dd x_1\wedge\dots\wedge\dd x_n\sum_{i=1}^n\left(1-\frac{px_i^2}{|x|^2}\right)\\
            &=\frac{1}{|x|^p}\dd x_1\wedge\dots\wedge\dd x_n\left(n-\frac{p(x_1^2+\dots+x_n^2)}{|x|^2}\right)\\
            &=\frac{n-p}{|x|^p}\dd x_1\wedge\dots\wedge\dd x_n
        \end{align*}
        \item For \(p=n, \dd\omega=0.\)
    \end{enumerate}

    \begin{proposition}
        Consider an oriented $k$-manifold-with-boundary and a form $\omega\in\Omega^k(M)$ supported on $c_1(I^k)\cap c_2(I^k)$
        \begin{equation}
            \int_{c_1}\omega=\int_{c_2}\omega=:\int_M\omega
        \end{equation}
    \end{proposition}
    Suppose $\omega\in\Omega^k(M),$ choose a partition of unity $\{\phi_i\}$ subordinate to the cover of $M$ by open sets that can be covered by ``good cubes''.
    Define \begin{equation}
        \int_M\omega=\sum_i\int_M\phi_i\omega
    \end{equation}
    Comments
    \begin{enumerate}
        \item If $M$ is compact, this always makes sense.
        \item In general, first define ``integrable'' forms as $\sum_i\int_M\phi_i|\omega|<\infty$. Note that $c^*(\phi_i\omega)=f\dd x_1\wedge\dots\wedge\dd x_k\in\Omega^k(I^k)$
    \end{enumerate}
    Important: This is independent of the partition of unity.
    \begin{equation}
        \int_M^{(\varphi_i)}=\sum_i\int\phi_i\omega=\sum_{i,j}\int\psi_j\phi_i\omega=\sum_j\int\psi_j\omega=\int_M^{(\psi_j)}\omega
    \end{equation}
    Notes
    \begin{enumerate}
        \item $\int_M\omega$ is linear in $\omega$.
        \item $\int_{-M}\omega=-\int_M\omega$, where $-M$ is the $M$ with the reversed orientation. (when you pull back with the cubes, let $c_1(x_1,\dots,x_k)=c_0(1-x_1,x_2,\dots,x_k))$ hence the transition function is $\phi(x_1,\dots,x_k)=(1-x_1,x_2,\dots,x_k)$. In the change of variables formula, the determinent of $\phi$ is $-1$.
    \end{enumerate}
    
    \begin{theorem}[Stokes' Theorem]
        If $M$ is a compacted and oriented $k$-manifold-with-boundary and $\omega\in\Omega^{k-1}(M)$, then 
        \begin{equation}
            \int_M\dd\omega=\int_{\partial M}\omega
        \end{equation}
        \begin{proof}
            Suppose we know the theorem on forms $\omega$ with $\mathrm{supp}\omega\subset\mathrm{im} c_i$ for a single cube $c_i$. Start with
            \begin{equation}
                \int_{\partial M}\omega
            \end{equation}
            Take an appropriate partition of unity for $M$, it is also a partition of unity for $\partial M$. Then,
            \begin{equation}
                \int_{\partial M}\omega=\sum_i\int_{\partial M}\varphi_i\omega=\sum_i\int_M\sum_i\int_M\dd(\phi_i\omega)=\sum_i\int_M\dd(\varphi_i)\wedge\omega+\int_{M}\varphi_i\dd\omega=\int_{M}\dd(\sum_i\varphi_i)\omega+\int_{\varphi_i}\dd\omega=\int_{M}\varphi_i\dd\omega
            \end{equation}
            To prove the theorem for small sets, consider two cases: 
            
            Case $I$ (interior): $\mathrm{supp}\omega\subset\mathrm{int}M$ and can be covered by a single interior cube. Here,
            \begin{equation}
                \int_{\partial M}\omega =0 \text{ as } \omega|_{\partial M}=0
            \end{equation}
            \begin{equation}
                \int_M\dd\omega=\int_{c_i}\dd\omega=\int_{I^k}c_i^*(\dd\omega)=\int_{I^k}\dd(c_i^*\omega)=\int_{\partial I^k}c_i^*\omega=\int_{c_{i*}(\partial I^k)}\omega=\int 0=0
            \end{equation}
            Case $B$ (boundary): $\mathrm{supp}\omega\cap\partial M\neq 0$\dots

            Choose $c_b(0,y,\dots,y_{k-1})$ intersects $\partial M$,
            \begin{equation}
                \int_M\dd\omega=\int_{\partial I^k}c_b^*\omega=\int_{\partial c_b}\omega=-\int_{(c_b)(1,0)}=-\int_{y_1\dots y_{k-1}}c_b(1,0)^*\omega=\int_{\partial M}\omega
            \end{equation}

        \end{proof}
    \end{theorem}

\end{document}
