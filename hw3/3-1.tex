\documentclass{exam}
\title{MAT257 PSET 3---Question 1}
\author{Jonah Chen}

\usepackage[utf8]{inputenc}
\usepackage[margin=0.5in]{geometry}

\usepackage{braket}
\usepackage{physoly}
\usepackage{currfile}
\usepackage{gensymb}
\usepackage{amssymb}
\usepackage{pgf,tikz,pgfplots}
\usepackage{mathrsfs}
\usepackage{textcomp}
\usepackage{parskip}
\setlength{\parindent}{0em}
\usetikzlibrary{arrows}
\numberwithin{equation}{section}
\pgfplotsset{compat=1.16}
\everymath{\displaystyle}
\newcommand{\R}{\mathbb{R}}
\begin{document}
    \sffamily
    \maketitle
    If $f:\R^n\to\R^m$ is differentiable at $a\in\R^n$, then by definition there exists a linear map $\lambda:\R^n\to\R^m$ where
    \begin{align*}
        \lim_{h\to 0}\frac{|f(a+h)-f(a)-\lambda h|}{|h|}=0
    \end{align*}
    Also, as $\lambda:\R^n\to\R^m$ is a linear map, $|\lambda h|\leq M|h|$ for some finite $M$. This means that 
    \begin{align*}
        \forall\varepsilon_1>0\exists\delta_1>0:|h|<\delta_1\implies|f(a+h)-f(a)-\lambda h|&<\varepsilon_1|h|\\
        \implies|f(a+h)-f(a)-\lambda h|&\leq|f(a+h)-f(a)|+|\lambda h|\\
        &\leq|f(a+h)-f(a)|+M|h|\\
        &<\varepsilon_1|h|\\
        \implies |f(a+h)-f(a)|<(&\varepsilon_1-M)|h|
    \end{align*}

    To show the continuity of $f$ at $a$, it is required that $\lim_{x\to a}f(x)=f(a)$. Define $h:=x-a$, so, $\lim_{h\to 0}f(a+h)=f(a)$. This means we need to show
    \begin{align*}
        \forall\varepsilon_2>0\,\exists\,\delta_2>0:|h|<\delta_2\implies |f(a+h)-f(a)|<\varepsilon_2
    \end{align*}

    As $f$ is differentiable, we can choose any $\varepsilon_1>0$ and there must exist $\delta_1>0$. So, choose $\varepsilon_1=M+\epsilon_2$


    
\end{document}