\documentclass{exam}
\title{MAT257 PSET 3---Question 4}
\author{Jonah Chen}

\usepackage[utf8]{inputenc}
\usepackage[margin=0.5in]{geometry}

\usepackage{braket}
\usepackage{physoly}
\usepackage{currfile}
\usepackage{gensymb}
\usepackage{amssymb}
\usepackage{pgf,tikz,pgfplots}
\usepackage{mathrsfs}
\usepackage{textcomp}
\usepackage{parskip}
\setlength{\parindent}{0em}
\usetikzlibrary{arrows}
\numberwithin{equation}{section}
\pgfplotsset{compat=1.16}
\everymath{\displaystyle}
\newcommand{\R}{\mathbb{R}}
\begin{document}
    \sffamily
    \maketitle
    \begin{enumerate}[label=(\alph*)]
        \item Consider the functions $f_1:\R^3\to\R^2, f_2:\R^2\to\R, f_3:\R\to\R$ defined by
        \begin{align*}
            f_1(x,y,z)&=(\log x , y)
            \\
            f_2(a,b)&=ab\\
            f_3(c)&=\exp(c)
        \end{align*}
        The derivative of $f_1$ is $f_1'=\begin{pmatrix}
            1/x & 0 & 0\\
            0 & 1 & 0
        \end{pmatrix}$ by Theorem 2-3(3)

        The derivative of $f_2$ is $f_2'=\begin{pmatrix}
            b & a
        \end{pmatrix}$ by Theorem 2-3(5)

        The derivative of $f_3$ is $f_3'=\exp(c)$.
        
        Note that $f_3(f_2(f_1(x,y,z)))=x^y=f(x,y,z)$. By chain rule,
        \begin{align*}
            f'(x,y,z)&=f_3'(f_2(f_1(x,y,z)))\cdot f_2'(f_1(x,y,z))\cdot f_1'(x,y,z)\\
            &=\begin{pmatrix}
                \exp(y\log x)
            \end{pmatrix}\begin{pmatrix}
                y & \log x
            \end{pmatrix}\begin{pmatrix}
                1/x & 0 & 0\\
                0 & 1 & 0
            \end{pmatrix}=\begin{pmatrix}
                x^{y-1}y & x^y\log x & 0
            \end{pmatrix}
        \end{align*}
        \item Now, $g(x,y,z)=(x^y, z)=(f(x,y,z),z)$, where $f(x,y,z)=x^y$ whose derivative is found in part (a). By Theorem 2-3(3),
        \begin{align*}
            g'(x,y,z)=\begin{pmatrix}
                f'(x,y,z)\\
                (z)'
            \end{pmatrix}=\begin{pmatrix}
                x^{y-1}y & x^y\log x & 0\\
                0 & 0 & 1
            \end{pmatrix}
        \end{align*}
        \item Consider the function $k:\R^3\to\R^3$ where $
            k=\begin{pmatrix}
                1 & 1 & 0\\
                0 & 0 & 1\\
                0 & 0 & 0
            \end{pmatrix}$
            Then, $k'=k$ by Theorem 2-3(2) as $k$ is a linear map. 
            
            Note that $h_1(x,y,z)=(x+y, z, 0)$. Therefore, 
            $h(x,y,z)=f(k(x,y,z))=(x+y)^z$. By chain rule,
            \begin{align*}
                h'(x,y,z)&=f'(k(x,y,z))\cdot k'(x,y,z)\\
                &=f'(x+y,z,0)\cdot k(x,y,z)\\
                &=\begin{pmatrix}
                    (x+y)^{z-1}z & (x+y)^z\log(x+y) & 0
                \end{pmatrix}\begin{pmatrix}
                    1 & 1 & 0\\
                0 & 0 & 1\\
                0 & 0 & 0
                \end{pmatrix}\\
                &=\begin{pmatrix}
                    (x+y)^{z-1}z &  (x+y)^{z-1}z & (x+y)^z\log(x+y)
                \end{pmatrix}
            \end{align*} 
    \end{enumerate}
\end{document}