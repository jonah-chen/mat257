\documentclass{exam}
\title{MAT257 PSET 9---Question 1}
\author{Jonah Chen}
\date{}
\usepackage[utf8]{inputenc}
\usepackage[margin=0.5in]{geometry}

\usepackage{braket}
\usepackage{physoly}
\usepackage{currfile}
\usepackage{gensymb}
\usepackage{amssymb}
\usepackage{pgf,tikz,pgfplots}
\usepackage{mathrsfs}
\usepackage{textcomp}
\usepackage{parskip}
\setlength{\parindent}{0em}
\usetikzlibrary{arrows}
\numberwithin{equation}{section}
\pgfplotsset{compat=1.16}
\everymath{\displaystyle}
\newcommand{\R}{\mathbb{R}}

\begin{document}
    \sffamily
    \maketitle
    First, we define the following $C^\infty$ building blocks from $\R\to\R$ which are used to construct a $C^\infty$ partition of unity. 
    \begin{align*}
        a(x)&=\begin{cases}
            0 & x\leq 0\\
            \exp(-1/x) & x>0
        \end{cases}\\
        b(x)&=a(x)a(1-x)\\
        c(x)&=\frac{\int_{0}^{x}b(t)\dd t}{\int_{0}^1b(t)\dd t}\\
        d(x)&=\frac{c(x)}{c(1-x)+c(x)}
    \end{align*}
    It is important to note a few properties of these building blocks. It is known that $a\in C^\infty$, so hence $b\in C^\infty$. 
    
    Also, $b$ is zero if $x\notin(0,1)$ and nonzero otherwise. Hence, $c\in C^\infty$ also as the denominator is nonzero, and $c$ is $0$ for $x\leq0$ and $1$ for $x\geq1$. 
    
    $d$ has this property also, but it is also important to note that $d(x)+d(1-x)=1$. $d$ is also $C^\infty$ because $c(x)$ and $c(1-x)$ can never add to zero.

    \begin{enumerate}[label=(\alph*)]
        \item Let
        \begin{align*}
            f(x)=\begin{cases}
                3(-1)^n/n & x\in[n+1/3,n+2/3]\land n\in\mathbb N\\
                0 & \text{otherwise}
            \end{cases}
        \end{align*}
        Clearly, $f$ vanishes for $x<1$ and for $x\in(n-1/3,n+1/3)$ for all positive integers $n$. For any $x\in[n+1/3,n+2/3]$, $f$ is a constant $3(-1)^n/n$. Hence, $\int_{n+1/3}^{n+2/3}f=(-1)^n/n$ as desired.

        A plot of $f$ would look like this:
        \begin{center}
            \begin{tikzpicture}
                \begin{axis}[xmin=-1,xmax=10,ymin=-3.1,ymax=1.6]
                    \addplot[color=red,domain=-1:1.3333]{0};
                    \addplot[color=red,domain=1.33333:1.66667]{-3};
                    \addplot[color=red,domain=1.66667:2.33333]{0};
                    \addplot[color=red,domain=2.33333:2.66667]{3/2};
                    \addplot[color=red,domain=2.66667:3.33333]{0};
                    \addplot[color=red,domain=3.33333:3.66667]{-1};
                    \addplot[color=red,domain=3.66667:4.33333]{0};
                    \addplot[color=red,domain=4.33333:4.66667]{3/4};
                    \addplot[color=red,domain=4.66667:5.33333]{0};
                    \addplot[color=red,domain=5.33333:5.66667]{-3/5};
                    \addplot[color=red,domain=5.66667:6.33333]{0};
                    \addplot[color=red,domain=6.33333:6.66667]{1/2};
                    \addplot[color=red,domain=6.66667:7.33333]{0};
                    \addplot[color=red,domain=7.33333:7.66667]{-3/7};
                    \addplot[color=red,domain=7.66667:8.33333]{0};
                    \addplot[color=red,domain=8.33333:8.66667]{3/8};
                    \addplot[color=red,domain=8.66667:9.33333]{0};
                    \addplot[color=red,domain=9.33333:9.66667]{-1/3};
                    \addplot[color=red,domain=9.66667:10]{0};
                \end{axis}
            \end{tikzpicture}
        \end{center}
        \newpage
        \item For integers $n$, define \begin{align*}
            U_n&:=(n-1/3,n+4/3)\\
            \varphi_n(x)&:=\begin{cases}
                d\left(\frac{3}{2}(x-n-1/3)\right) & x\leq n+1/3\\
                1 & n+1/3<x<n+2/3\\
                d\left(-\frac{3}{2}(x-n-2/3)\right) & x\geq n+2/3
            \end{cases}
        \end{align*}
        Then, choose the open cover $\mathcal{U}=\{U_n\}$ of $\R$. This clearly covers $\R$ because and $x\in\R$ is contained in $U_{\lfloor x\rfloor}$.
        
        Choose a $C^\infty$ partition of unity $\Phi=\{\varphi_n\}$. $\varphi_i$ is subordinate to $U_i$, and on $U_i$ only $\varphi_{i-1},\varphi_i,\varphi_{i+1}$ can be nonzero and they sum to $1$ due to the properties of $d$. Hence, $\Phi$ is locally finite. Thus, the integral
        \begin{align*}
            \int_\R f=\sum_{n=-\infty}^\infty\int\varphi_nf
        \end{align*}
        Note that for $n\leq0$, the integral is zero because $f$ vanishes. For $n>0$, $f$ is only nonzero on $x\in[n+1/3,n+2/3]$ in the support of $\varphi_n$. On the set $[n+1/3,n+2/3]$, $\varphi_n=1$ Thus,
        \begin{align*}
            \int\varphi_nf=\int_{n+1/3}^{n+2/3}f=\int_{n+1/3}^{n+2/3}3(-1)^n/n=\frac{(-1)^n}{n}
        \end{align*}
        Then,
        \begin{align*}
            \sum_{n=-\infty}^\infty\int\varphi_nf=\sum_{n=1}^\infty\frac{(-1)^n}{n}
        \end{align*}
        This is some alternating version of the harmonic series, which is known to conditionally converge. As the series does not absolutely converge, the integral does not exist.

        \item Define
        \begin{align*}
            W_n&=U_{2n}\cup U_{2n+1}\\
            V_n&=U_{4n}\cup U_{4n+2}\cup U_{2n+1}\\
            \phi_n&=\varphi_{2n}+\varphi_{2n+1}\\
            \psi_n&=\varphi_{4n}+\varphi_{4n+2}+\varphi_{2n+1}\\
        \end{align*}
        Choose the open cover $\mathcal{W}=\{W_n\}$ and $\mathcal{V}=\{V_n\}$. These are clearly still open covers of $\R$ because they contain all the sets in the open cover $\mathcal{U}$. 
        
        Then, $\Phi=\{\phi_n\}$ and $\Psi=\{\psi_n\}$ are partitions of unity subordinate to $\mathcal{W}$ and $\mathcal{V}$ respectively. Similar to part (b), $\Phi$ and $\Psi$ are locally finite as at most three functions can be nonzero on each set of $U_n$ and $V_n$. They are also subordinate to their respective open covers and sum to $1$ on each $x\in\R$.
        
        Taking the ``integral'' using $\Phi$,
        \begin{align*}
            \sum_{n=-\infty}^\infty\int\phi_nf=-1+\sum_{n=1}^\infty\frac{1}{2n}-\frac{1}{2n+1}=-1+\sum_{n=1}^\infty\frac{1}{2n+4n^2}=-\log2
        \end{align*}
        The infinite sum converges absolutely as it converges and all terms are positive. The sum was computed with mathematica.
        
        Taking the ``integral'' using $\Psi$,
        \begin{align*}
            \sum_{n=-\infty}^\infty\int\psi_nf=-1+\frac{1}{2}+\sum_{n=1}^\infty\frac{1}{4n}+\frac{1}{4n+2}-\frac{1}{2n+1}=\sum_{n=1}^\infty\frac{1}{4 n + 8 n^2}=-\frac{\log 2}{2}
        \end{align*}

        The infinite sum converges absolutely as it converges and all terms are positive. The sum was computed with mathematica.

        Clearly these two sums for the two partitions of unity are not equal.
    \end{enumerate}
\end{document}