\documentclass{exam}
\title{MAT257 PSET 4---Question 3}
\author{Jonah Chen}

\usepackage[utf8]{inputenc}
\usepackage[margin=0.5in]{geometry}

\usepackage{braket}
\usepackage{physoly}
\usepackage{currfile}
\usepackage{gensymb}
\usepackage{amssymb}
\usepackage{pgf,tikz,pgfplots}
\usepackage{mathrsfs}
\usepackage{textcomp}
\usepackage{parskip}
\setlength{\parindent}{0em}
\usetikzlibrary{arrows}
\numberwithin{equation}{section}
\pgfplotsset{compat=1.16}
\everymath{\displaystyle}
\newcommand{\R}{\mathbb{R}}
\begin{document}
    \sffamily
    \maketitle
    \begin{enumerate}[label=(\alph*)]
        \item Note that the integral of $g_1$ does not depend on the second variable $y$. Therefore, applying the single variable fundamental theorem of calculus on $D_2f(x,y)=\partial_y\int_0^yg_2(x,t)\dd t=g_2(x,y)$
        \item Currently, $D_1f(x,y)=g_1(x,0)$ by the single variable fundamental theorem of calculus. Changing $f$ so that 
        \begin{align*}
            f(x,y)=\int_0^xg_1(t,y)\dd t+\int_0^yg_2(x,t)\dd t
        \end{align*}
        will change $D_1f(x,y)$ to $g_1(x,y)$.
        \item $f_d(x,y)=\frac{x^2+y^2}{2}$
        \item $f_c(x,y)=xy$
    \end{enumerate}
\end{document}