\documentclass{exam}
\title{MAT257 PSET 4---Question 4}
\author{Jonah Chen}

\usepackage[utf8]{inputenc}
\usepackage[margin=0.5in]{geometry}

\usepackage{braket}
\usepackage{physoly}
\usepackage{currfile}
\usepackage{gensymb}
\usepackage{amssymb}
\usepackage{pgf,tikz,pgfplots}
\usepackage{mathrsfs}
\usepackage{textcomp}
\usepackage{parskip}
\setlength{\parindent}{0em}
\usetikzlibrary{arrows}
\numberwithin{equation}{section}
\pgfplotsset{compat=1.16}
\everymath{\displaystyle}
\newcommand{\R}{\mathbb{R}}
\begin{document}
    \sffamily
    \maketitle
    \begin{enumerate}[label=(\alph*)]
        \item Note that $a+te_i=\begin{pmatrix}
            a_1 & \dots & a_i+t & \dots & a_n
        \end{pmatrix}$. Then, 
        \begin{align*}
            \lim_{t\to0}\frac{f(a+te_i)-f(a)}{t}=\lim_{t\to0}\frac{f(a_1,\dots,a_i+t,\dots,a_n)-f(a_1,\dots,a_n)}{t}=D_if(a)
        \end{align*}
        by definition on Spivak pp.25.
        \item Using the definition of the directional derivative
        \begin{align*}
            D_{x}f(a)=\lim_{h\to0}\frac{f(a+xh)-f(a)}{h}
        \end{align*}
        Consider making a change of variables $k:=h/t$ for $t\neq0$ (the direction derivative with respect to the 0 vector does not make sense). Note that as $h\to 0$, $k\to 0$, and $h$ can be written as $tk$. Rewriting the limit results in 
        \begin{align*}
            D_{x}f(a)=\lim_{k\to0}\frac{f(a+xtk)-f(a)}{tk}=\frac{1}{t}\lim_{k\to0}\frac{f(a+xtk)-f(a)}{k}=\frac{1}{t}D_{tx}f(a)
        \end{align*}
        Multiplying both sides by $t$ yields $D_{tx}f(a)=tD_xf(a)$.
        \item As $f$ is differentiable at $a$, \begin{align*}
            \lim_{h\to0}\frac{|f(a+h)-f(a)-Df(a)h|}{|h|}=0
        \end{align*}
        Make the substitution $h=tx$ for nonzero vector $x$. As $h\to0,t\to0$. Then,
        \begin{align*}
            \lim_{t\to0}\frac{|f(a+tx)-f(a)-tDf(a)x|}{|t|}=0
        \end{align*}
        Rearranging the definition of the directional derivative, $\lim_{t\to0}\frac{|f(a+tx)-f(a)-tD_xf(a)|}{|t|}=0$. Since the derivative is unique, $Df(a)x=D_xf(a)$. 

        Since $Df(a)$ is a linear operator, $Df(a)(x+y)=Df(a)(x)+Df(a)(y)$. So, $D_{x+y}f(a)=D_xf(a)+D_yf(a)$. 
    \end{enumerate}
\end{document}