\documentclass{exam}
\title{MAT257 PSET 4---Question 1}
\author{Jonah Chen}

\usepackage[utf8]{inputenc}
\usepackage[margin=0.5in]{geometry}

\usepackage{braket}
\usepackage{physoly}
\usepackage{currfile}
\usepackage{gensymb}
\usepackage{amssymb}
\usepackage{pgf,tikz,pgfplots}
\usepackage{mathrsfs}
\usepackage{textcomp}
\usepackage{parskip}
\setlength{\parindent}{0em}
\usetikzlibrary{arrows}
\numberwithin{equation}{section}
\pgfplotsset{compat=1.16}
\everymath{\displaystyle}
\newcommand{\R}{\mathbb{R}}
\begin{document}
    \sffamily
    \maketitle
    \begin{enumerate}[label=(\alph*)]
        \item By 1-variable Fundamental Theorem of Calculus,
        \begin{align*}
            \partial_x f(x,y)&=g(x+y)\\
            \partial_y f(x,y)&=g(x+y)
        \end{align*}
        \item $\partial_xf(x,y)=g(x)$. Note that $\int_y^xg=-\int_x^yg$. Hence, $\partial_yf(x,y)=-g(y)$.
        \item Define $h(u)=\int_a^ug$ and $k(x,y)=xy$. Then, $f(x,y)=h(k(x,y))$ By chain rule, 
        \begin{align*}
            \partial_xf(x,y)&=h'(k(x,y))\cdot\partial_xk=g(xy)\cdot y\\
            \partial_yf(x,y)&=h'(k(x,y))\cdot\partial_yk=g(xy)\cdot x
        \end{align*}
        \item As $f$ does not depend on $x$, $\partial_xf=0$. For the $y$ partial derivative, note that $f(x,y)=h(h(y))$ so $\partial_yf(x,y)=(h\circ h)'(y)$. By chain rule (single variable), \begin{align*}
            \partial_yf(x,y)=(h\circ h)'(y)=h'(h(y))\cdot h'(y)=g(h(y))\cdot g(y)=g\left(\int_{b}^yg\right)\cdot g(y)
        \end{align*}
    \end{enumerate}
\end{document}