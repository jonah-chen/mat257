\documentclass{exam}
\title{MAT257 PSET 4---Question 6}
\author{Jonah Chen}

\usepackage[utf8]{inputenc}
\usepackage[margin=0.5in]{geometry}

\usepackage{braket}
\usepackage{physoly}
\usepackage{currfile}
\usepackage{gensymb}
\usepackage{amssymb}
\usepackage{pgf,tikz,pgfplots}
\usepackage{mathrsfs}
\usepackage{textcomp}
\usepackage{parskip}
\setlength{\parindent}{0em}
\usetikzlibrary{arrows}
\numberwithin{equation}{section}
\pgfplotsset{compat=1.16}
\everymath{\displaystyle}
\newcommand{\R}{\mathbb{R}}
\begin{document}
    \sffamily
    \maketitle
    Consider $h_x(t)=f(tx)$. Since $f$ is differentiable, taking the derivative of $h_x$ is the same as taking the partial derivative of $f$ with respect to $t$, as $x$ in $h_x$ is constant. Using chain rule for $f$,
    $h_x'(t)=\partial_tf(tx)=f'(tx)\cdot x$. 
    
    Next, note that by one variable fundamental theorem of calculus $\int_{0}^1h_x'(t)\dd t=h(1)-h(0)=f(x)-f(0)=f(x)$ as $f(0)=0$.
    
    Also, $f(x)=\int_0^1h_x'(t)\dd t=\int_0^1f'(tx)\cdot x\dd t=\sum_{i=1}^nx_i\int_0^1Df_i(tx)\dd t$. So, define $g_i(x):=\int_0^1Df_i(tx)\dd t$, then $f(x)=\sum_{i=1}^nx_ig_i(x)$.
\end{document}