\documentclass{exam}
\title{MAT257 PSET 18---Question 4}
\author{Jonah Chen}
\date{}
\usepackage[utf8]{inputenc}
\usepackage[margin=0.5in]{geometry}

\usepackage{braket}
\usepackage{physoly}
\usepackage{currfile}
\usepackage{gensymb}
\usepackage{amssymb}
\usepackage{pgf,tikz,pgfplots}
\usepackage{mathrsfs}
\usepackage{textcomp}
\usepackage{parskip}
\usepackage{bbm}
\setlength{\parindent}{0em}
\usetikzlibrary{arrows}
\pgfplotsset{compat=1.16}
\everymath{\displaystyle}
\newcommand{\R}{\mathbb{R}}

\begin{document}
    \sffamily
    \maketitle
    Given a $k+l+1$ manifold $M$, a $k$-form $\omega$ on $M$ and a $l$-form $\eta$ on $M$. Consider integrating the $k+l$-form $\omega\wedge\eta$. Since $M$ is compact and oriented, Stokes theorem holds. Furthermore, $M$ has no boundary so the integral over $\partial M$ is 0.
    \begin{align*}
    0=\int_{\partial M}\omega\wedge\eta=\int_{M}\dd(\omega\wedge\eta)=\int_M(\dd\omega\wedge\eta+(-1)^k\omega\wedge\dd\eta)
    \end{align*}
    Since integration is linear,
    \begin{align*}
    \int_M\dd\omega\wedge\eta=(-1)^{k+1}\int_M\omega\wedge\dd\eta.
    \end{align*}
    The sign for integration by parts is $s=(-1)^{k+1}$.
\end{document}